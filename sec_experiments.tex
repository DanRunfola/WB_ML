\section{experiments}



\subsection{Simulation Experiments}



\subsection{World Bank aid data}

In this subsection, we examine an application of these methods to examine the research question: "How effective have the environmental safeguards of World Bank projects been in preventing deforestation?".  
These safeguards include environmental education and impact assessment programs, reforestation activities, and other environmental protection activities (Nielson and Tierney 2003; Ledec and Posas 2003; Quintero 2007). 
To examine the impact these activities have had on forest cover, we seek to isolate the causal impact of any given World Bank project on the forest cover - measured using satellite data - which it is proximate to.  
\par
In this paper, we specifically examine the benefits which causal trees have for examining causal impacts at a global scale.  
Because World Bank projects exist in highly heterogeneous environments (see figure X), traditional causal methods which identify a single effect across the entire sample - without identifying impacts unique to meaningful sub populations - are insufficient.  
In this approach, we illustrate that the causal tree enables an examination of impact on sub-populations within a given dataset, without the a-priori definition of those sub-populations.
We pose that it is a more effective and accurate way to understand the geographically varied impacts of World Bank projects.
\par

Our method is based on the R package rpart. Rpart support user defined split function, therefor, we can use the split criterion function \ref{eq:5}. To improve the efficiency of the r program, we use rcpp and call C++ functions inside the split and evaluation function for each node in the tree. To further improve the c++ functions, we use openmp inside the C++ functions. 