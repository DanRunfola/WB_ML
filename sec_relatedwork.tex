\section{related work}
Causality \cite{Pearl:2000:CMR:331969} plays an important role in many area. In this paper, we focus on the heterogeneous causal effects.
Some paper in the literature use tree based machine learning technique to estimate heterogeneous causal effects.
In \cite{journals/jmlr/SuTWNL09}, they use statistical test as the criterion for node splitting.  In \cite{1504.01132}, they use causal trees to estimate heterogeneous treatment effect. However, they do not show what if in some nodes, there is only treated units or only untreated units and then how to estimate the heterogeneous causal effects.\\
Some paper use forest based machine learning technique to estimate heterogeneous causal effects. In \cite{1510.04342}, they use casual forest to do heterogeneous causal effects estimation, and they share the same idea in paper \cite{Denil:2014} that they use difference data for the structure of the tree and the estimation value within each node.