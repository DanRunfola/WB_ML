
%%%%%%%%%%%%%%%%%%%%%%% file typeinst.tex %%%%%%%%%%%%%%%%%%%%%%%%%
%
% This is the LaTeX source for the instructions to authors using
% the LaTeX document class 'llncs.cls' for contributions to
% the Lecture Notes in Computer Sciences series.
% http://www.springer.com/lncs       Springer Heidelberg 2006/05/04
%
% It may be used as a template for your own input - copy it
% to a new file with a new name and use it as the basis
% for your article.
%
% NB: the document class 'llncs' has its own and detailed documentation, see
% ftp://ftp.springer.de/data/pubftp/pub/tex/latex/llncs/latex2e/llncsdoc.pdf
%
%%%%%%%%%%%%%%%%%%%%%%%%%%%%%%%%%%%%%%%%%%%%%%%%%%%%%%%%%%%%%%%%%%%


\documentclass[runningheads,a4paper]{llncs}

\usepackage{amssymb}
\usepackage{subfig}
\usepackage{amssymb}
\usepackage{pdfpages}
\usepackage{graphicx}
\usepackage{caption}
\setcounter{tocdepth}{3}
\usepackage{graphicx}
\usepackage{caption}
%\usepackage{subcaption}

\usepackage{url}
\urldef{\mailsa}\path|{alfred.hofmann, ursula.barth, ingrid.haas, frank.holzwarth,|
\urldef{\mailsb}\path|anna.kramer, leonie.kunz, christine.reiss, nicole.sator,|
\urldef{\mailsc}\path|erika.siebert-cole, peter.strasser, lncs}@springer.com|    
\newcommand{\keywords}[1]{\par\addvspace\baselineskip
\noindent\keywordname\enspace\ignorespaces#1}

\begin{document}

\mainmatter  % start of an individual contribution

% first the title is needed
%\title{Machine Learning Method for Estimating Heterogeneous Causal Effects of the World Bank Projects}
\title{Quantifying Heterogeneous Causal Treatment Effects in World Bank Aid Projects}
% a short form should be given in case it is too long for the running head
\titlerunning{Quantifying Heterogeneous Causal Treatment Effects}

% the name(s) of the author(s) follow(s) next
%
% NB: Chinese authors should write their first names(s) in front of
% their surnames. This ensures that the names appear correctly in
% the running heads and the author index.
%
\author{Jianing Zhao\inst{1} \and Daniel M. Runfola\inst{2} \and Peter Kemper\inst{1} }
%\thanks{Please note that the LNCS Editorial assumes that all authors have used
%the western naming convention, with given names preceding surnames. This determines
%the structure of the names in the running heads and the author index.}%
%\and Ursula Barth\and Ingrid Haas\and Frank Holzwarth\and\\
%Anna Kramer\and Leonie Kunz\and Christine Rei\ss\and\\
%Nicole Sator\and Erika Siebert-Cole\and Peter Stra\ss er}
%
\authorrunning{Quantifying Heterogeneous Causal Treatment Effects}
% (feature abused for this document to repeat the title also on left hand pages)

% the affiliations are given next; don't give your e-mail address
% unless you accept that it will be published
\institute{College of William and Mary, Williamsburg, VA 23187-8795, USA\\
\email{\{jzhao,kemper\}@cs.wm.edu}
\and
AidData, 427 Scotland Street, Williamsburg, VA. 23185 USA \\
\email{drunfola@aiddata.org}
}

%
% NB: a more complex sample for affiliations and the mapping to the
% corresponding authors can be found in the file "llncs.dem"
% (search for the string "\mainmatter" where a contribution starts).
% "llncs.dem" accompanies the document class "llncs.cls".
%

\toctitle{Quantifying Heterogeneous Causal Treatment Effects}
\tocauthor{Heterogeneous Treatment Effects}
\maketitle

%Jianing's abstract
  
 %In technical terms, we are interested in estimating heterogeneity in causal effects and conduct inference about the magnitude of the differences in project effects across subsets of world bank projects. To do so, we analyze a data set with data on world bank projects from 1982 to 2014 that contains project characteristics, geographical, and environmental data such as temperature and precipitation.
% as well as spatial and spatially related information.  
%The key challenge for this analysis is that the observational data does not allow us to directly measure the difference between the result of performing a project and of not performing a project as reality only gives us the choice to do one of the two.
%Following recent research results by Athey and Imbens, we employ a combination of machine learning techniques such as random forests with techniques from causal inference to measure the average treatment effect, i.e. the average effect of a project, for subsets of geographic locations. We validate our findings with project evaluations from the world bank and outcomes of competing econometric models.

\begin{abstract}

The World Bank provides hundreds of millions of dollars in development finance to countries across the world every year.  In order to ensure these funds are being spent as effectively as possible, there is a natural drive to promote better understandings of what projects work and which don't.  However, the global extent of these projects results in a great deal of heterogeneity in impacts due to geographic, cultural, and other factors.  Recent research by Athey and Imbens has illustrated the potential for hybrid machine learning and causal inferential techniques which may be able to capture such heterogeneity.  We apply their approach using a geolocated dataset of World Bank projects, and augment this data with satellite-retrieved characteristics of their geographic context (including temperature, precipitation, slope, distance to urban areas, and many others).  We use this information in conjunction with causal tree(CT), transformed outcome tree (TOT), and random forest with TOT trees approaches to (a) segment the data into relevant 'control' and 'treatment' groups, and (b) examine the impact of World Bank projects on vegetative cover.  We contrast our findings with project evaluations from the World Bank, and outcomes of more traditional, empirical econometric models.
\end{abstract}




\section{Introduction}
% memo
% worldbank website: http://www.worldbank.org/en/about/what-we-do
%For any serious human activity, there is the natural question: what difference does it make? 
%If one leaves the realm of scientific lab experiments, the challenge is that the ?ground truth? for a causal effect is not
%observed for any individual unit: we observe the unit with the treatment, or without the treatment, but not both at the same time,
%
%For a large scale, worldwide operation like the world bank, this question is particularly challenging to answer. 
%The world bank's overarching goal is in economic development;  it formulates this as to "end extreme poverty by decreasing the percentage of people living on less than \$1.90 a day to no more than 3\%"  and to "promote shared prosperity by fostering the income growth of the bottom 40\% for every country" \cite{worldbankwebsite}.
%Each individual project that it funds extends over time, so what is an appropriate measure of difference over time?
%Each individual project is located in space, so what is an appropriate measure of difference in space?
%
%Some notes
%\begin{itemize}
%\item general problem in development: project has a time, space, and economic dimension, 
%\item how to measure success
%\item how to measure what's going on
%\item how to measure impact (and when), how to infer causality
%\item problem present in particular in aid projects for third world
%\item describe data
%\item formulate research question that is addressed 
%\item countries that have bad causal or good effects across different project starting years for the 5 continents
%\item figure show aid project in the world map
%
%\item main area or countries that have good or bad effects
%\end{itemize}

\begin{figure}
	\centering
	\includegraphics[width=\textwidth]{figs/wbprojects.png}
	\caption{world bank projects} \label{fig:wbprojects}
\end{figure}

For any serious human activity, there is the natural question: what difference does it make? 
Identifying and quantifying causal effects from data is one of the most interesting  research problem across many disciplines. 
For example, this arises in measuring the effectiveness of a drug in medical studies, in measuring the impact of changes in an e-commerce website design on customers, in evaluating the effectiveness of public policies. In our case, the world bank is interested in measuring the impact of aid projects it funded and supported all over the world over 30 years. 

The world bank's overarching goal is in economic development;  it formulates this as to "end extreme poverty by decreasing the percentage of people living on less than \$1.90 a day to no more than 3\%"  and to "promote shared prosperity by fostering the income growth of the bottom 40\% for every country" \cite{worldbankwebsite}. At an abstract level, the world bank provides funding for aid projects and can rely on a metric of choice to measure differences before a project begins and after a project ends. 
At the level of an individual project, we face the general crux of all observational studies, i.e., we can not observe the exact same geographic, environmental, social, economic, and historical setting with or without the project. So for measuring a difference, one has to rely on making meaningful comparisons between locations that are sufficiently similar. In addition to that, the world bank's operation is large scale with a large number of projects and worldwide, which creates a huge variability in the specific kind of project, the project's size, location, socio-economic, environmental, and historical setting. Figure \ref{fig:wbprojects} shows the locations of a set of world bank projects with 1168 projects in 16415 locations  that were performed between 2001 and 2012 on a global map.

The research questions, we investigate in this paper are:
\begin{itemize}
\item Can we estimate the impact of a project?
\item Can we identify subsets of entities in our data set that are meaningful to compare?
\item Can we identify attributes that are indicative of projects that show a positive (or negative) impact?
\end{itemize}

To do so, we enhance a given data set of world bank projects with additional information about the geographic, environmental, and economic 
characteristics over a number of years and rely on state-of-the-art techniques to estimate heterogenous causal effects. 
In our case, it is not interesting to estimate the overall average effect of all aid projects, but to identify subsets of projects by attributes and estimate average effects for individual subsets.
%Instead of investigate the causal effects for the whole population, in this paper, we are interested in estimating heterogeneous causal effects for subpopulations by features or covariates. We can estimate heterogeneity by covariates on causal effects and then conducting inference for a distinct unit.\\
%To avoid getting extreme treatment effects which lead to a spurious heterogeneous result, in disciplines such as clinical trial, they use pre-planed subgroup to analyze, for economic, they have pre-analysis plans for randomized experiments. With a data driven approach, the advantage is to discover some other causal effects instead of only the pre-planed subgroups.\\
To estimate heterogenous causal effects, there are several candidates, such as classification and regression trees  \cite{Breiman:2001:RF:570181.570182}, random forests \cite{breiman1984classification}, LASSO \cite{Tibshirani94regressionshrinkage}, and support vector machines (SVM) \cite{Vapnik1998}. 
%In this paper, we use the regression tree, the other methods such as random forest is also good candidate, but we focus on the regression tree in this paper.\\
In this paper, we follow the work of Athey and Imbens \cite{1504.01132} who demonstrated how regression trees and in conclusion also random forests can be adjusted to estimate heterogenous causal effects. It is based on the Rubin Causal Model or potential outcome framework where causal effects are comparisons between observed outcomes and counterfactual outcomes one would have observed under the absence of an aid project \cite{Imbens:2015:CIS:2764565}. Regression trees and random forests in traditional machine learning rely on training with data with known outcomes. Athey and Imbens showed that one can estimate the conditional average treatment effect on a  subset with regressions trees after an appropriate data transformation using propensity scores. This leads to the notion of transformed-outcome trees and causal trees that we use for our analysis. 

The rest of the paper is structured as follows. In Section 2, we recall the basic methodology for the calculation of transformed-outcome trees, causal trees, and random forests.  Section 3 introduces the data set, its characteristics, preprocessing steps and the calculation of propensity scores. In Section 4, we present the outcome of the analysis. We conclude in Section 5.
%In tradition, we can use decision trees to do prediction using the trained data or labeled data. We can build the regression tree to predicting the causal effects with the features as nodes in the tree. However, for the causal inference, the challenge is we do not have such data,in rubin causal model \cite{Imbens:2015:CIS:2764565}, we can only have the treated data or untreated data, but not both at the same time, hence we do not know the ground truth for prediction. We can't follow the traditional supervised machine learning method that we contract the tree with the trained the data and then use the the test data to do prediction based on the constructed tree. Follow the work of Athey and Imbens \cite{1504.01132}, we use causal tree to do heterogeneous causal effects estimation. However, in practice,for example in our case the world bank data set, within a node, there maybe only treated or untreated data, we will discuss in the paper how to explain such data and other issues \\
%\begin{itemize}
%\item first analysis on heterogeneous causal effect of world bank aid projects using machine learning method
%\item 
%
%\end{itemize}

\section{methodology}

\subsection{conditional average treatment effects}

Suppose we have a data set with n iid units with $ i = 1, \cdots, n$, for each unit, it has a feature vector $X_{i}  \in [0,1]^d$, a response $Y_{i} \in \mathbb{R}$ and treatment indicator $W_{i} \in \{0,1\}$.\\
For unit level causal effect, we can use Rubin causal model to estimate the average causal effect as shown in function \ref{eq:1},
\begin{equation} \label{eq:1}
\tau_{i} = Y_{i}(1) - Y_{i}(0) \\
\end{equation}
In this paper, we are interested in heterogenous causal effect as \ref{eq:2}, this estimator is proposed by \cite{RePEc:ecm:emetrp:v:71:y:2003:i:4:p:1161-1189},
\begin{equation} \label{eq:2}
\tau(x) = \mathbb{E} \big[{Y_{i}(1) - Y_{i}(0) \mid X_{i} = x}\big]\\
\end{equation}
The challenge is we know either $Y_{i}(0)$ or $Y_{i}(1)$, but not both at the same time. 
We need to make a unconfoundness assumptions to estimate $\tau(x)$.
\begin{equation} \label{eq:3}
W_{i} \perp \left( Y_{i}(1), Y_{i}(0) \right)  \mid X_{i}\\
\end{equation}
Under the unconfoundness assumption, we can get the causal effect as 

\begin{equation}\label{eq:3.5}
\tau(x) = \mathbb{E} \big[   Y^{*} \mid X_{i} =  x\big] 
\end{equation}
where $Y^{*}$  is function \ref{eq:4}, $e(x)$ is function \ref{eq:4.5}, to estimate the propensity score, there are several ways for calculation such as \cite{rose:rubi:cent:1983}, \cite{HoImaKin07}, in this paper, we use logic regression to calculate the pscore.
\begin{equation} \label{eq:4}
Y_{i}^{*} =  Y_{i}^{obs} \cdot \frac{W_{i} - e(X_{i})}{e(X_{i}) \cdot (1 - e(X_{i}))} \\
\end{equation}

\begin{equation}\label{eq:4.5}
e(x) = \mathbb{E} \big[  W_{i} \mid X_{i} =  x \big]
\end{equation}

\subsection{Causal Tree Model}
We use regression tree to estimate the heterogeneous causal effects, the first step is to construct the tree. To construct the regression tree, we recursively partition the node until the size of the node is less than a threshold we set or the gain of split is negative.\\
In classic regression tree, mean square error (MSE) is often used to as the criterion for node splitting, the average value within the node is used as the estimator. Following Asthey and Imbens \cite{1504.01132}, we use \ref{eq:5} as the estimator and we calculate the error of the node by summing $Y_{i} - \hat{\tau}(X_{i})$.
\begin{equation} \label{eq:5}
\begin{split}
\hat{\tau}^{CT}(X_{i}) =  \frac{\sum_{i:X_{i} \in \mathbb{X}_{l}}  Y_{i}^{obs} \cdot W_{i} / \hat{e}(X_{i})  }{\sum_{i:X_{i} \in \mathbb{X}_{l}}  W_{i} / \hat{e}(X_{i})}  \\
 - \frac{\sum_{i:X_{i} \in \mathbb{X}_{l}}  Y_{i}^{obs} \cdot (1 - W_{i}) / (1 - \hat{e}(X_{i}))  }{\sum_{i:X_{i} \in \mathbb{X}_{l}}  (1 - W_{i}) / (1 - \hat{e}(X_{i}))}
 \end{split}
\end{equation}

\subsection{Pruning the tree}

To avoid overfitting of the tree, we need to prune the tree. We use the minimal cost complexity pruning and we define it as \ref{eq:6}. $\alpha$ is the complexity parameter, with it we can construct the regression with the right size.
\begin{equation}\label{eq:6}
R_{\alpha}(T) = R(T) + \alpha \left|\tilde{T}\right|
\end{equation}
where R(T) is the resubstitution error estimate of tree T, $\left|\tilde{T}\right|$ is defined as the complexity of the tree, which is the number of leaves in the tree,\\
To estimate the error of a node, we use function \ref{eq:7},
\begin{equation}\label{eq:7}
R(t) = \sum_{i=1}^{N}(Y_{i} - \hat{\tau}(X_{i}))
\end{equation}
where N is the total units in the nodes. \\

To get a sequence of $\alpha$, we minimize function \ref{eq:8},
\begin{equation}\label{eq:8}
g(t,T) = \frac{R(t) - R(T_{t})}{\left|\tilde{T_{t}}\right| - 1}
\end{equation}
where $T_{t}$ is a subtree of T rooted at node t.\\
We use the weakest link cutting to determine $\alpha$ and use it as the complexity parameter when we build the tree for the whole data set. 

Because the tree structure is not stable, we use V-fold cross validation to estimate the errors for different $\alpha$.
\begin{equation}\label{eq:9}
R(T(\alpha)) = \frac{1}{N}\sum_{1}^{N}Err(\alpha;Y_{i}, X_{i})
\end{equation}
where $Err(\alpha;Y_{i}, X_{i})$  is the errors for construction the tree using $Y_{i}, X_{i}$ data with parameter $\alpha$. 















\section{data}

\subsection{data sources and collection}
We leverage the following data sources in this analysis:
\begin{center}
  \begin{tabular}{| l | c | c |}
    \hline
    Variables &  Description & Source \\ \hline
    $Forest Cover$ & NASA Long Term Data Record measurements of vegetative cover & http://ltdr.nascom.nasa.gov/cgi-bin/ltdr/ltdrPage.cgi \\ \hline
    $World Bank Project Locations$ & Double-blind geocoded information on the geographic location of each World Bank project & http://aiddata.org/level1/geocoded/worldbank \\ \hline
    $Distance to Rivers$ & The calculated average distance to all rivers & http://hydrosheds.cr.usgs.gov/index.php \\ \hline
    $Distance to Commercial Rivers$ & Calculated average distance to all commercial rivers & http://hydrosheds.cr.usgs.gov/index.php  \\ \hline
    $Distance to Roads$ &  Distance to nearest road & http://sedac.ciesin.columbia.edu/data/set/groads-global-roads-open-access-v1\\ \hline
    $Elevation$ & Elevation data measured from the Shuttle Radar Topography Mission &http://www2.jpl.nasa.gov/srtm/\\ \hline
    $Slope$ &  Slope data calculated based on the Shuttle Radar Topography Mission & http://www2.jpl.nasa.gov/srtm/\\ \hline
    $Accesibility to Urban Areas$ & European Commission Joint Research Centre estimation of urban travel times. & http://forobs.jrc.ec.europa.eu/products/gam/download.php\\ \hline
    $Population Density$ & Center for International Earth Science estimation of population density, derived from Nighttime Lights & http://sedac.ciesin.columbia.edu/data/collection/gpw-v3  \\ \hline
    $Air Temperature$ & University of Delware Long term, global temperature data interpolated from weather station measurements. &http://climate.geog.udel.edu/~climate/ \\ \hline
    $Precipitation$ & University of Delware Long term, global precipitation data interpolated from weather station measurements.&http://climate.geog.udel.edu/~climate/\\ 
    \hline
    
  \end{tabular}
\end{center}

    
\subsection{data pre-processing}
  This analysis uses three key types of data: satellite data to measure vegetation, data on the geospatial locations of World Bank projects, and covariate datasets (the sources of which are detailed above). 
Our primary variable of interest is the fluctuation of vegetation proximate to World Bank projects, which is derived from long-term satellite data (NASA 2015). 
There are many different approaches to using satellite data to approximate vegetation on a global scale, and satellites have been taking imagery that can be used for this purpose for over three decades.  
Of these approaches, the most frequently used is the Normalized Difference Vegetation Index (NDVI).  The NDVI is a metric that has been used since the early 1970s, and is one of the simplest and most frequently used approaches to approximating vegetative biomass.  
NDVI measures the relative absorption and reflectance of red and near-infrared light from plants to quantify vegetation on a scale of -1 to 1, with vegetated areas falling between ~0.2 and 1. 
The reflectance by chlorophyll is correlated with plant health, and multiple studies have illustrated that it is generally also correlated with plant biomass. 
In other words, healthy vegetation and high plant biomass tend to result in high NDVI values (Dunbar 2009).  
Using NDVI as an outcome measure has a number of other benefits, including the long and consistent time periods for which it has been calculated.  
While the NDVI does have a number of challenges - including a propensity to saturate over densely vegetated regions, the potential for atmospheric noise (including clouds) to incorrectly offset values, and reflectances from bright soils providing misleading estimates - the popularity of this measurement has led to a number of improvements over time to offset many of these errors.  
This is especially true of measurements from longer-term satellite records, such as those used in this analysis, produced from the MODIS and AVHRR satellite platforms (NASA 2015).
\par
The second primary dataset used in this analysis measures where - geographically - World Bank projects were located.  This dataset was produced by AidData (2016), relying on a double-blind coding system where two experts employ a defined hierarchy of geographic terms and independently assign uniform latitude and longitude coordinates, precision codes, and standardized place names to each geographic feature. If the two code rounds disagree, the project is moved into an arbitration round where a geocoding project manager reconciles the codes to assign a master set of geocodes for all of the locations described in the available project documentation. This approach also captures geographic information at several levels—coordinate, city, and administrative divisions—for each location, thereby allowing the data to be visualized and analyzed in different ways depending upon the geographic unit of interest. Once geographic features are assigned coordinates, coders specify a precision code that varies from 1 (exact point) to 9 (national-level project or program).
AidData performs many procedures to ensure data quality, including de-duplication of projects and locations, correcting logical inconsistencies (e.g. making sure project start and end dates are in proper order), finding and correcting field and data type mismatches, correcting and aligning geocodes and project locations within country and administrative boundaries, validating place names and correcting gazetteer inconsistencies, deflating financial values to constant dollars across projects and years (where appropriate), strict version control of intermediate and draft data products, semantic versioning to delineate major and minor versions of various geocoded datasets, and final review by a multidisciplinary working group. 

\section{experiments}



\subsection{Simulation Experiments}



\subsection{World Bank aid data}

In this subsection, we examine an application of these methods to examine the research question: "How effective have the environmental safeguards of World Bank projects been in preventing deforestation?".  
These safeguards include environmental education and impact assessment programs, reforestation activities, and other environmental protection activities (Nielson and Tierney 2003; Ledec and Posas 2003; Quintero 2007). 
To examine the impact these activities have had on forest cover, we seek to isolate the causal impact of any given World Bank project on the forest cover - measured using satellite data - which it is proximate to.  
\par
In this paper, we specifically examine the benefits which causal trees have for examining causal impacts at a global scale.  
Because World Bank projects exist in highly heterogeneous environments (see figure X), traditional causal methods which identify a single effect across the entire sample - without identifying impacts unique to meaningful sub populations - are insufficient.  
In this approach, we illustrate that the causal tree enables an examination of impact on sub-populations within a given dataset, without the a-priori definition of those sub-populations.
We pose that it is a more effective and accurate way to understand the geographically varied impacts of World Bank projects.
\par

Our method is based on the R package rpart. Rpart support user defined split function, therefor, we can use the split criterion function \ref{eq:5}. To improve the efficiency of the r program, we use rcpp and call C++ functions inside the split and evaluation function for each node in the tree. To further improve the c++ functions, we use openmp inside the C++ functions. 
\section{related work}
Causality \cite{Pearl:2000:CMR:331969} plays an important role in many area. In this paper, we focus on the heterogeneous causal effects.
Some paper in the literature use tree based machine learning technique to estimate heterogeneous causal effects.
In \cite{journals/jmlr/SuTWNL09}, they use statistical test as the criterion for node splitting.  In \cite{1504.01132}, they use causal trees to estimate heterogeneous treatment effect. However, they do not show what if in some nodes, there is only treated units or only untreated units and then how to estimate the heterogeneous causal effects.\\
Some paper use forest based machine learning technique to estimate heterogeneous causal effects. In \cite{1510.04342}, they use casual forest to do heterogeneous causal effects estimation, and they share the same idea in paper \cite{Denil:2014} that they use difference data for the structure of the tree and the estimation value within each node.

\section{Conclusions}
issues: number of control units in leaves

precision of answers

robustness of random forest for TOT

spill over effect in spatial data (big project next to untreated case)

spatial diversity in leaf nodes (not just diversity in values)
\bibliographystyle{splncs03}
\bibliography{bibtex}

\end{document}
