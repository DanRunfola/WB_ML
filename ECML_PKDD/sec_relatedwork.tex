\section{related work}
Causality \cite{Pearl:2000:CMR:331969} plays an important role in many area. In this paper, we focus on the heterogeneous causal effects.
Some paper in the literature use tree based machine learning technique to estimate heterogeneous causal effects.
In \cite{journals/jmlr/SuTWNL09}, they use statistical test as the criterion for node splitting.  In \cite{1504.01132}, they use causal trees to estimate heterogeneous treatment effect. However, they do not show what if in some nodes, there is only treated units or only untreated units and then how to estimate the heterogeneous causal effects.\\
Some paper use forest based machine learning technique to estimate heterogeneous causal effects. In \cite{1510.04342}, they use casual forest to do heterogeneous causal effects estimation, and they share the same idea in paper \cite{Denil:2014} that they use difference data for the structure of the tree and the estimation value within each node.\\
In \cite{1412.8563}, they change the item image size on ebay and observer the treatment effect as how much money people spent during the experiment. The difference between their work and ours is that they only change one factor, however, for aid data, a project may change several factor which is more complex compared to IT data. 
%\cite{Geurts00investigationand}, 
\cite{hirano2003efficient}
Regression random forest can give estimation of the conditional means, in \cite{Meinshausen:2006:QRF:1248547.1248582}, they use quantile regression  forest to estimate the distribution of the result instead of mean and they prove the algorithm is consistent. 
As discussed in  \cite{Biau:2012:ARF:2503308.2343682}, \cite{Wager:2014:CIR:2627435.2638587}, , \cite{Denil:2014}, there is a gap between theory property and practical use of random forest. 