\section{introduction}

Draw causal effects from data is one of the most interesting  research problem across many disciplines. For example, people want to know the effects of a drug in medical studies, companies would like to know the effect of their advertisement on customers, government seeks to evaluate the effect of public policies, for our case, world bank wants to know the effect of the aid projects they investigate around the world over 30 years. \\
Instead of investigate the causal effects for the whole population, in this paper, we are interested in estimating heterogeneous causal effects for subpopulations by features or covariates. We can estimate heterogeneity by covariates on causal effects and then conducting inference for a distinct unit.\\
To avoid getting extreme treatment effects which lead to a spurious heterogeneous result, in disciplines such as clinical trial, they use pre-planed subgroup to analyze, for economic, they have pre-analysis plans for randomized experiments. With a data driven approach, the advantage is to discover some other causal effects instead of only the pre-planed subgroups.\\
To estimate heterogenous causal effects, there are several candidates, for example, classification and regression tree  \cite{Breiman:2001:RF:570181.570182}, random forest \cite{breiman1984classification}, LASSO \cite{Tibshirani94regressionshrinkage}, SVM \cite{Vapnik1998} and so on. In this paper, we use the regression tree, the other methods such as random forest is also good candidate, but we focus on the regression tree in this paper.\\
In tradition, we can use decision trees to do prediction using the trained data or labeled data. We can build the regression tree to predicting the causal effects with the features as nodes in the tree. However, for the causal inference, the challenge is we do not have such data,in rubin causal model \cite{Imbens:2015:CIS:2764565}, we can only have the treated data or untreated data, but not both at the same time, hence we do not know the ground truth for prediction. We can't follow the traditional supervised machine learning method that we contract the tree with the trained the data and then use the the test data to do prediction based on the constructed tree. Follow the work of Athey and Imbens \cite{1504.01132}, we use causal tree to do heterogeneous causal effects estimation. However, in practice,for example in our case the world bank data set, within a node, there maybe only treated or untreated data, we will discuss in the paper how to explain such data and other issues \\

